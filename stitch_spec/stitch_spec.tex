\chapter{Calculating a stitched spectrum from time series}
\label{App:stitch_spec}

Many existing programs and scripts exist that compute the fast Fourier transform (FFT) and corresponding spectral density of an input time series. However, for a number of reasons we have found it advantageous to compute several FFT's from the time capture and to ``stitch'' them together to represent the spectrum. In this Appendix, we review our algorithm to compute the FFT's and to stitch them into a single spectrum.

Given an input time series, sampled at a frequency $1/\Delta t$ and consisting of $N$ points of amplitude $y_j$ ($j \in \{1, 2, 3, \ldots\}$), the FFT is defined as
\begin{align}\label{eqn:defFFT}
Y_k = \sum_{j=1}^N y_j \omega_N^{(j-1)(k-1)},
\end{align}
where $\omega_N = \exp[-2\pi i/N]$ is an $N^\text{th}$ root of unity. In general, the FFT algorithm works best when $N$ is a power of 2; if it is not, $y_j$ is usually padded with zeros until its length is a power of 2. Because $y_j \in \mathbb{R}$, $Y_k$ is symmetric ($Y_{k} = Y_{N-k+2}^*$ for $k \ge 2$) and the power spectral density $S_k$ can be written as
\begin{align}
S_k =
  \begin{cases}
   \frac{\Delta t}{N} |Y_1|^2 &\text{if } k = 1 \\
   2\frac{\Delta t}{N} |Y_k|^2 &\text{if } k \in \{2, 3, 4, \ldots N/2 + 1 \}
  \end{cases},
\end{align}
which is defined at frequencies $f_k = (k-1)/N \Delta t$, where $k \in \{1, 2, 3, \ldots N/2 + 1 \}$.

One can compute a single FFT of the entire time capture ($N = \Ntc$). However, this technique leads to a spectrum that appears extremely noisy; that is, the difference between the spectral density at adjacent frequencies is of the order of the spectral density itself, regardless of the length of the time capture. This property can be understood as follows. At any particular frequency, $S_k$ of a random time series follows a Chi-squared distribution with two degrees of freedom ($S_k \sim \chi_2^2$). A property of the $\chi_2^2$ distribution is that the variance $\sigma^2$ is equal to its mean $\mu$ squared. Therefore, $\sigma^2(S_k) = \mu^2(S_k)$ so that the spread of spectral densities is relatively large and the spectrum appears noisy. A longer time capture increases $\Ntc$, which in turn increases the frequency resolution---the width of the frequency bins $df = 1/N\Delta t = 1/\Ntc\Delta t$---but do not decrease $\sigma^2(S_k)$.

To reduce the apparent noise in the spectrum, it is typical to introduce some form of averaging. A popular method is to split the time capture into $\Navg$ equivalent-length segments, computing the FFT for each segment and averaging the spectral densities to form a single averaged spectrum. In other words, $S_k = (1/\Navg) \sum S_{k,n}$, where $S_k$ is now an averaged spectral density and $S_{k,n}$ is the spectral density at $f_k$ of the $n^\text{th}$ time segment. In this case, $S_k$ obeys the central limit theorem and, for sufficiently large $\Navg$, is normally distributed with a variance that scales as $1/\Navg$. Since $df = \Navg/\Ntc\Delta t$, we see that there is a tradeoff between high frequency resolution and low variance of the spectral density. Because the lowest frequency of the spectrum (zero frequency excluded) is equal to $df$, the tradeoff is equivalently between a spectrum that extends low in frequency and one that is highly averaged.

To circumvent this tradeoff, we observe that, for our application, a high frequency resolution is not necessary at high frequencies. Similarly, at low frequencies we can tolerate relatively few averages, which in turn gives us high frequency resolution. This situation suggests that we compute an averaged FFT with relatively low $\Navg_\text{,min}$ for low frequencies and large $\Navg_\text{,max}$ for high frequencies. The spectra generated by using different values of $\Navg$ can be stitched together at some intermediate frequency $\fcut$. If $\Navg_\text{,min}$ and $\Navg_\text{,max}$ are significantly different, intermediate values of $\Navg$ can be used as well. Our standard 1~hr time captures consist of $3200*1024 = 3,276,800$ points, sampled at 1024~Hz. To generate a spectrum, we use $\Navg = [3, 12, 50, 200, 800, 3200]$ to generate 6~FFT's, which we stitch together at the frequencies $\fcut = [1/2^{10}, 1/2^8, 1/2^6, 1/2^4, 1/2^2]*(512~\text{Hz}/50) = [0.01, 0.04, 0.16, 0.64, 2.56]$~Hz.

[@@@ Show some figures?]

[@@@ Advantages when fitting?]