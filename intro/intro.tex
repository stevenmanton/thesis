% This sample file is dedicated to the public domain.
\chapter{Introduction}
\label{chap:intro}

Abstract.

\section{Introduction}



\section{Symmetric and antisymmetric critical current noise} \label{Sec:sym_asym_Ic_noise}

[The following needs a lot of work! The only point is that asymmetric variations can't be much larger than symmetric variations.]

[Newer version (outline form):]

Suppose that a single junction has a noise spectrum $S_{I_0}$. Then symmetric critical current variations of the SQUID will have noise spectrum $S_{I_c,\text{sym}} = 2S_{I_0}$. There's a factor of 2 because the noise between junctions is uncorrelated. While asymmetric critical current variations do not directly change the critical current, they can indirectly change the critical current by inducing a flux. Asymmetric variations induce a circulating current that can, through the self-inductance of the SQUID, induce a flux. [What is the magnitude of the circulating current?? Isn't it just $S_{I_0}$?]
\begin{align}\label{blah}
S_{I_c,\text{asym}} = S_{I_0} L^2 \left(\frac{\partial I_c}{\partial \Phi}\right)^2
\end{align}
Next, because $|\partial I_c/\partial \Phi|_\text{max} \sim I_c/\Phi_0$ and $\beta_L = L I_c/\Phi_0$,
\begin{align}\label{blah}
S_{I_c,\text{asym,max}} \sim S_{I_0} (L I_c/\Phi_0)^2 = S_{I_0} \beta_L^2 \lesssim S_{I_0}.
\end{align}
Therefore, critical current noise due to asymmetric variations will never be much larger than that due to symmetric variations.

[Old version:]

The symmetric component of critical current noise $S_{I_c,\text{sym}}$ is independent of flux bias. However, because of the antisymmetric component $S_{I_c,\text{asym}}$, which couples to the SQUID as a flux and therefore adds noise proportional the the flux sensitivity of the SQUID, the magnitude of the noise can vary with flux bias:
\begin{align}\label{fig:experimental:SIasym}
\frac{\partial\Iloop}{\partial I_{c,\text{asym}}} = L^2 %
\left(\frac{\partial \Iloop}{\partial \Phi_m}\right)^2,
\end{align}
where $L$ is the self-inductance of the SQUID. Because the maximum $\partial\Iloop/\partial_m$ varies roughly as $I_c/\Phi_0$ for our devices, which have $\beta_L \equiv LI_c/\Phi_0 \lesssim 0.5$, Eq.~\eqref{fig:experimental:SIasym} can be approximated as
\begin{align}\label{fig:experimental:SIasym}
\left(\frac{\partial\Iloop}{\partial I_{c,\text{asym}}}\right)_{\text{max}} \approx %
\left(\frac{\beta_L\Phi_0}{I_c}\right)^2 %
\left(\frac{I_c}{\Phi_0}\right)^2 %
\approx \beta_L^2.
\end{align}
Because critical current variations within the junctions are independent, $S_{I_c,\text{sym}} = S_{I_c,\text{asym}}$. Therefore, we see that the maximum noise contribution due to asymmetric critical current variations is of the same order of magnitude as, or much lower than, that of symmetric variations. Furthermore, because the magnitude of $S_{I_c,\text{asym}}$ varies with SQUID inductance, which is often difficult to measure, it is difficult to know accurately.

\section*{Acknowledgments}

Too bad nobody helped me with the introduction either.
